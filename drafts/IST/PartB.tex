\documentclass[a4paper,11pt]{report}
\usepackage[latin1]{inputenc}
\usepackage{vmargin}
 
%% Marges
\setmarginsrb{2cm}{1cm}{2cm}{2cm}{1cm}{1cm}{1cm}{1cm}
%1 est la marge gauche
%2 est la marge en haut
%3 est la marge droite
%4 est la marge en bas
%5 fixe la hauteur de l'ent�te
%6 fixe la distance entre l'ent�te et le texte
%7 fixe la hauteur du pied de page
%8 fixe la distance entre le texte et le pied de page
\setlength{\marginparwidth}{0cm}
\setlength{\marginparsep}{0cm}
%\setcounter{tocdepth}{1}

%% XXX
\title{Proposal full title \\ Proposal Acronym}
\date{16/10/2001}
\begin{document}
\chapter*{Notice}
\begin{verbatim}
IST2001 - IV.3.3 Free software development: towards critical mass 
Objectives:

1.To foster in Europe a critical mass of development of free software released
  under GPL-compatible licenses. 
2.To make available European based support services for free software projects.

Focus:

1.Support services for free software developers at all stages of the project 
  lifecycle: project hosting certification, release, repositories, dissemination. 
2.Trials of free software development in the following fields: media technology 
  for personal users,co-operative information production and sharing, and usability. 
3.Socio-economic studies on technology assessment, economic assessment and derived 
  business models 

Types of actions addressed: Trials and studies Accompanying Measures. 
Links with WP 2000: Further focused Action Lines IST 2000 - IV.3.4 and IV.3.5 

more info : http://www.cordis.lu/ist/bwp_en5.htm
\end{verbatim}
\maketitle
\tableofcontents
\chapter*{B3. Objectives}
\textit{This section, which should not exceed two pages, describes the
scientific/technological objectives of the proposal. They should be achiebable
within the project, not through subsequent development, and should be stated in
a measurable and verifiable form. The Progess of the Take-Up actionwill be
measured against theses objectves in later reviews and assessments.\\
This section should clearly specify in addition to the technical objectioves -
also the quantified business objectives and how they are measured}
\\ \\
to be written
\chapter*{B4. Contribution to programme/Key action objectives}
\textit{This section, which should not exceed more than one page, describes how
the proposed Take-Up Action will contribute to the objectives of the programme
and Key action. This can be done by describing how the proposal meets the
objectives and focus of the Action line which it addresses}
\\ \\
to be written
\chapter*{B5. Innovation}
\textit{This secition, not exceeding two pages, describes the innovative
aspects of the proposed project. \\
Progress beyond the state of current practices within a user organisation
should be described. The individual action steps must be described in the
context of the real business case. A separate description of this underlying
baseline project/target application - in terms of technical and business
parameters - must be included. Choice and justfication for each of the new
practices including the supporting technologies (with comparison of not chosen
options) should be given (e.g. new methodologies, technologies, hardware and/or
software tools, subcontracted experts etc. -- whatever is applicable for the
experiment.}
\\ \\
to be written.
\chapter*{B6. Project workplan}
\textit{This section consisely describes the work planned to achieve the
objectives of the proposed Take-Up Action. The recommended length, excluding
the forms specified below, is up to 10 pages. An introduction should explain
the structure of th workplan and how the workplan must be broken into
workpackages (WPs) which should follow the logical phases of the projects'
life, and include management of the project and assessment of progress and
results. \\ \\
There must be a separate WP each for 'measureement and evaluation of results}
\\ \\ not yet written
\end{document}
