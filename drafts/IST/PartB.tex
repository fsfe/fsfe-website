\documentclass[a4paper,11pt]{report}
\usepackage[latin1]{inputenc}
\usepackage{vmargin}
 
%% Marges
\setmarginsrb{2cm}{1cm}{2cm}{2cm}{1cm}{1cm}{1cm}{1cm}
%1 est la marge gauche
%2 est la marge en haut
%3 est la marge droite
%4 est la marge en bas
%5 fixe la hauteur de l'ent�te
%6 fixe la distance entre l'ent�te et le texte
%7 fixe la hauteur du pied de page
%8 fixe la distance entre le texte et le pied de page
\setlength{\marginparwidth}{0cm}
\setlength{\marginparsep}{0cm}
%\setcounter{tocdepth}{1}

%% XXX
\title{Proposal full title \\ Proposal Acronym}
\date{16/10/2001}
\begin{document}
\chapter*{Notice}
\begin{verbatim}
IST2001 - IV.3.3 Free software development: towards critical mass 
Objectives:

1.To foster in Europe a critical mass of development of free software released
  under GPL-compatible licenses. 
2.To make available European based support services for free software projects.

Focus:

1.Support services for free software developers at all stages of the project 
  lifecycle: project hosting certification, release, repositories, dissemination. 
2.Trials of free software development in the following fields: media technology 
  for personal users,co-operative information production and sharing, and usability. 
3.Socio-economic studies on technology assessment, economic assessment and derived 
  business models 

Types of actions addressed: Trials and studies Accompanying Measures. 
Links with WP 2000: Further focused Action Lines IST 2000 - IV.3.4 and IV.3.5 

more info : http://www.cordis.lu/ist/bwp_en5.htm
\end{verbatim}
\maketitle
\tableofcontents
\chapter*{some text}
%%%%%%%%%%%%%%%%%%%

\subsubsection{Mission}
Although hosting facilities are widely used in the freesoftware community the
existing solutions have some drawbacks, that refrain their wide adoption as
well in the Libre software community than in the software industry. \\
Due to a quick design, they are hard to distribute. They mainly are a front end
to a group of dedicated machines that provides a precise set of
functionnalities. Even if the acces is remote, all data is centralized as one
site physically hosts an huge number of projects. This is bad practice in terms
of data safety, data access and scalability. \\
Also due to the number of services and their inter-relations, it is not simple
to set up a new site that provides hosting. This is a real brake to a large
number of sites using this technology.  There is no complete solutions to
install easily any of the technologies described above. \\
They rely too much on managing procedures and human control. For example, a
site needs to be managed by people that acknowledge project creation. \\
At last the most active effort to build such a sytem has recently become
proprietary.  Therefore the produced work is not available. 
%%%%%%%%%%%%%%%%%%%%%%%%%%%
\chapter*{B3. Objectives}
\addcontentsline{toc}{chapter}{B3. Objectives}
\textit{This section, which should not exceed two pages, describes the
scientific/technological objectives of the proposal. They should be achievable
within the project, not through subsequent development, and should be stated in
a measurable and verifiable form. The Progress of the Take-Up action will be
measured against theses objectives in later reviews and assessments.\\
This section should clearly specify in addition to the technical objectives -
also the quantified business objectives and how they are measured}
\\ \\
\section*{Decentralized.}
Any given machine running NAME can host a list of
projects (with pointers to the machines hosting the
projects), read-only copies of projects (with pointers
to the machines hosting the read-write instance of the
project), read-write projects.
\\
The read-only projects are mirrored from the machine
hosting the read-write instance of the project or other
machines hosting a read-only copy. The format used is
based on XML and defined by CoopX (http://coopx.eu.org).
\\
The number of read-only and read-write projects hosted
on a given machine is controlled by the maintainer of the
machine, depending on the available resources. For instance,
an average machine with lots of bandwidth could host at most
one hundred read-write projects and ten thousand read-only
projects.
\\
At a given time, only one machine hosts a read-write
project. It is the responsibility of the project
maintainers to ensure this.  Should a machine become
unavailable for some reason, the administrators of the
project can toggle the read-only flag of one of the
read-only mirrors and start using it as a
replacement. 
\\
Such a decentralized setup is not the most efficient
replication system but it is definitely simple and
requires very little effort.
The first objective is to build a code base with all tools that current hosting
facilities have. The first audit we made shows there are a Bug tracking system,
per project mailing lists and web forums, project management tools

\section*{Multi machine.}
A single machine may not be powerful enough for someone
willing to host a large number of projects. Planning a
multi machine setup in advance is mostly impossible
because many strategies can be implemented depending on
the needs. For instance a site like SourceForge will not
implement the same strategies than a site like Geocities
because the usage pattern is not the same.
\\
A Debian package repository will be used to store the
packages that implement various strategies. They can all
coexist together. Let's say, for instance, that the
package NAME-single implements a single machine setup
and depends on exim, cvs and bind. There can also be a
NAME-multi-cvs, NAME-multi-exim, NAME-multi-bind,
each installed on a separate machine and that configure
the corresponding service for a multi-machine setup.


\section*{Packaging.}

Every bit is packaged for Debian. Although it may be
desirable to package it for other GNU/Linux distributions,
this is not our goal. The cooperative nature of Debian and
the package dependency management are essential. When the
set of packages reach a stable state it will be possible
to contribute them to Debian without a need to find an
agreement with a company. The dependency management allows
to consistently use a large set of packages.
\\
The packages will be produced in a source specific to the
project. This source (currently
http://france.fsfeurope.org/debian/) contains three types
of packages: copy of existing Debian packages slightly modified
for the needs of NAME, packages of unfinished software written
for NAME, packages whose sole purpose is to bind a set of 
packages together and configure them. 
\\
The modified copy of existing Debian packages are only stored
while waiting for the Debian maintainer or the upstream of the
software to integrate the change. It is never meant to be 
permanent. Communication is always established with the Debian
maintainer and the upstream before the modified copy is built.
Should we fail to do that, NAME will become a concurrent Debian
distribution and this is to be avoided at all cost.
\\
The software specific to NAME is packaged at a very
early stage. Basically the package is done whenever the
software performs at least one task. Properly packaging a
software takes a lot of time and failing to do it from the
very beginning is asking for troubles. It also allows
developpers to use the software and contribute to it in an
organized way. Finally, since NAME is a collection of
many packages it allows to debug and architecture their
interdepencies while writing code.

\section*{Templates and internationalization. }

User visible content is stored in tempates. The preferred
format for HTML pages is XHTML. 
\\
Every string displayed is stored using gettext files and
obeys the system locale.

\section*{Upgradable.}
	    
Each release provides an upgrade procedure from the
previous version. The upgrade may imply a planned
down-time, NAME is commited to provide an upgrade
facility that would work without stopping the service. The
XML dump of the database will be used to upgrade the data
using XSL files. The database structure itself is upgraded
with a set of SQL orders. The associated services (smtp,
cvs, etc.) are reset to their initial state (empty) and
the backend scripts are run on the upgraded database to
restore the desired state. 

\section*{Export/Import using CoopX.}

CoopX provides a set of XML based schemas to represent all
the data related to a project. This includes the description
of a user (P3P), contact information (XMLvCard) and custom
made schemas for the CVS tree and the bug tracker information
for instance. 
\\
The goal of CoopX is not to promote a standard but to use
them where possible. The bug tracker information, for
instance, is codified using a DTD with some documentation
to understand it.  There are no plans to submit it to W3C
or other similar organizations.
\\
The CoopX format is used for upgrades and for communication
between distinct NAME platforms.

\chapter*{B4. Contribution to program/Key action objectives}
\textit{This section, which should not exceed more than one page, describes how
the proposed Take-Up Action will contribute to the objectives of the program
and Key action. This can be done by describing how the proposal meets the
objectives and focus of the Action line which it addresses}
\\ \\
to be written
\chapter*{B5. Innovation}
\addcontentsline{toc}{chapter}{B5. Innovation}
\textit{This section, not exceeding two pages, describes the innovative
aspects of the proposed project. \\
Progress beyond the state of current practices within a user organization
should be described. The individual action steps must be described in the
context of the real business case. A separate description of this underlying
baseline project/target application - in terms of technical and business
parameters - must be included. Choice and justification for each of the new
practices including the supporting technologies (with comparison of not chosen
options) should be given (e.g. new methodologies, technologies, hardware and/or
software tools, subcontracted experts etc. -- whatever is applicable for the
experiment.}
\\ \\
%%% just cut & paste for now
\subsection*{Context}
Most of free or open source software is collaboratively developped using an
hosting plateform. They are mostly web sites. The frontend integrates different
functionnalities that are available to project developpers and administrators
and linked them with each other. The backend is mainly implemented with
scripts. \\
Current hosting plaforms are implemented as a  web site that provides
functionnalities to developpers. The most common ones include  users and groups
of users management, project repository (usualy based on Concurent Versionning
System), per project web pages and document sharing, per project mailing lists
and web forums, bugtracking system, project management system (audit, task
management), global site management and information feeding. There is currently
four main web hosting facilities : SourceForge (http://sourceforge.net/),
Savannah (http://savannah.gnu.org/), VHFFS (http://tuxfamily.org/), nd
Picolibre (http://picolibre.eu.org/). In terms of number of users and projects,
Savannah and SourceForge are the leaders.
\\
\chapter*{B6. Project workplan}
\addcontentsline{toc}{chapter}{B6. Project Workplan}
\textit{This section concisely describes the work planned to achieve the
objectives of the proposed Take-Up Action. The recommended length, excluding
the forms specified below, is up to 10 pages. An introduction should explain
the structure of the workplan and how the workplan must be broken into
workpackages (WPs) which should follow the logical phases of the projects'
life, and include management of the project and assessment of progress and
results. \\ \\
There must be a separate WP each for 'measurement and evaluation of results}
\\ \\ not yet written
\end{document}
