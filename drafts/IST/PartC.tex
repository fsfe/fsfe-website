\documentclass[a4paper,11pt]{report}
\usepackage[latin1]{inputenc}
\usepackage{vmargin}
 
%% Marges
\setmarginsrb{2cm}{1cm}{2cm}{2cm}{1cm}{1cm}{1cm}{1cm}
%1 est la marge gauche
%2 est la marge en haut
%3 est la marge droite
%4 est la marge en bas
%5 fixe la hauteur de l'ent�te
%6 fixe la distance entre l'ent�te et le texte
%7 fixe la hauteur du pied de page
%8 fixe la distance entre le texte et le pied de page
\setlength{\marginparwidth}{0cm}
\setlength{\marginparsep}{0cm}
%\setcounter{tocdepth}{1}

%% XXX
\title{Proposal full title \\ Proposal Acronym}
\date{16/10/2001}
\begin{document}
\chapter*{Notice}
\begin{verbatim}
IST2001 - IV.3.3 Free software development: towards critical mass 
Objectives:

1.To foster in Europe a critical mass of development of free software released
  under GPL-compatible licenses. 
2.To make available European based support services for free software projects.

Focus:

1.Support services for free software developers at all stages of the project 
  lifecycle: project hosting certification, release, repositories, dissemination. 
2.Trials of free software development in the following fields: media technology 
  for personal users,co-operative information production and sharing, and usability. 
3.Socio-economic studies on technology assessment, economic assessment and derived 
  business models 

Types of actions addressed: Trials and studies Accompanying Measures. 
Links with WP 2000: Further focused Action Lines IST 2000 - IV.3.4 and IV.3.5 

more info : http://www.cordis.lu/ist/bwp_en5.htm
\end{verbatim}
\maketitle
\tableofcontents
\chapter*{C3. Community added value and contribution to EC policies}
\addcontentsline{toc}{chapter}{C3. Community added value and contribution to EC policies}
\textit{This section, which should not exceed two pages, should identify which
issue at the European level the proposal is addressing. It should also describe
how the experience will be transferred to other organisations across Europe. \\
This section must include a description of the European dimension of their
action, in terms of profile of the wider community which will benefit from the
result of their action. This will help define the European level issue
addressed by the proposal. \\
Participants should be prepared to be grouped under a Support Node or Cluster,
to ensure that those not directly involved can benefit from the results. In
this way, the Take-Up Actions will generate European added value, beyond the
limit achievable through individual projects.}
\\ \\
to be written
\chapter*{C4. Contribution to Community social objectives}
\addcontentsline{toc}{chapter}{C4. Contribution to Community social objectives}
\textit{This section, which should not exceed two pages, should identify which
issue at the European level the proposal is addressing. It should also describe
how the experience will be transferred to other organisations across Europe. \\
This section must include a description of the European dimension of their
action, in terms of profile of the wider community which will benefit from the
result of their action. This will help define the European level issue
addressed by the proposal. \\
Participants should be prepared to be grouped under a Support Node or Cluster,
to ensure that those not directly involved can benefit from the results. In
this way, the Take-Up Actions will generate European added value, beyond the
limit achievable through individual projects.}
\\ \\
to be written
\
\textit{This section, not exceeding two pages, should describe how the proposed
subject wil contribute to meeting the social objectives of the Community, and
should focus on the contribution of the project to improving employment
prospects and the use and development of skills in Europe. \\
Where aplicable the descriptions should also cover how the proposed measure
contributes to improving quality of life and health and safety (including
working condition), and/or to preserving or enhancing the environment and
natural ressources. This might include the contribution of the proposed work to
meeting relevant regulatory requirements.}
\\ \\
to be written
\chapter*{C5. Management}
\addcontentsline{toc}{chapter}{C5. Management}
\textit{This section, not exceeding two pages, should describe how the Take-Up
Action will be managed, the decision making structures to be applied, the
communication flow within the consortium and the quality assurance measures
which will be implemented, and how legal and ethical obligations, for example
concerning encryption or security of personal data, will be met.
\\
The clustering of Assesment project will allow effective co-ordination, early
agreement - between users and suppliers - on globally accepted assesment
metrics and on competitive specifications, the timely delivery of relevant and
measurable results of assesment and their dissemination to other users
worldwide.\\
Actions management of an action is linked to the management of the underlying
business case (target application, baseline project. Relations and dependencies
including the risks involved should also be described. Beging linked under a
Support Node or Cluster will require co-operation with other actions.}
\\ \\
to be written.
\chapter*{C6. Description of the consortium}
\addcontentsline{toc}{chapter}{C6. Description of the consortium}
\textit{One page description of the consortium stating whom the participants
are, what their roles and functions in the consortium are, and how they
complement each other.\\
Where subcontractors are involved they should be clearly identified, and the
scope and nature of their work needs to be justified}
\\ \\ not yet written
\chapter*{C7. Description of the participants}
\addcontentsline{toc}{chapter}{C7. Descriptions of the participants}
\textit{Short description of the participating organisations including (no more
than two pages per organisation):
The expertise and experience of the organisation,
Short CVs of the key persones to be involved indicating relevant exprerience,
expertise and involvement in other EC projects. (Each CV no more than 10
lines). \\
This section should include a brief description of the main business of the
proposer(s) (or consortium, if applicable), including their business motivation
for the proposed action, and the business relevance. \\
It should be clear from this section what the user's current practices are,
which improvements are introduced during the course of the action, and what the
required and expected business practice upon completion of the action is}.
\\ \\ 
%% ALCOVE DE
\section*{Daniel Riek - Alcove Deutschland CTO}
Daniel Riek switched to GNU/Linux in late 1993 short after graduating
from school. In early 1997 during his studies of computing science at the
university of Bonn he founded the Linux service company ID-PRO together
with some friends. During the next years he worked as a developer and
engineer, later he served in the management area of ID-PRO. After ID-PRO
received a major venture capital financing, Daniel became the chairman
and speaker of the board of directors. There he was responsible for the
technical strategy, representation, the community and public relationship.
He also was, together with the companies CTO, responsible for the technical
support and the software development activities of the company.
Caused by the capital market crisis in late 2000 ID-PRO failed to acquire
new funds, that would have been necessary due to the venture capital driven
strategy. After a short rest Daniel joined Alcove to continue his work in
the Free Software business. Daniel is a strong supporter of the Free
Software Foundation Europe and is a member of the board of germany's Linux
vendors association Linux-Verband LIVE e.V.
\\
Next contributors : Christian, Jaime.
\chapter*{C8. Economic development and scientific and technological prospects}
\addcontentsline{toc}{chapter}{C8. Economic development and scientific and
technological prospecs}
\textit{This section, which should not exceed three pages, should describe
plans for the dissemination and - if relevant - the exploitation of the results
for the consortium as a whole and for the individual participants in contrete
terms. \\
This section must include an explanation of the potential for - internal and
external - replication, redeployment and/or reuse of results of the action. It
should a) describe in detail how the anticipated results will be used
internally within the participating organisations and what decisions may be
taken upon successful completion of the action, and b) identify the business
communities, sharing the problem to be solved by the action and describe a plan
how the results achieved, benefits obtained and lessons learnt (including key
competencies acquired, deficiencies removed, lasting changes) will be
disseminated/communicated and transferred to those communities. \\
Given the real benefits of an action may only become apparent several months
after completion of the action, proposers should indicate expected benefits in
the longer term (including measures that will be taken to assess them)}
\\ \\
to be written
\end{document}
