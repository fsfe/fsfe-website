\documentclass[a4paper,11pt]{report}
\usepackage[latin1]{inputenc}
\begin{document}
\section*{Notice}
This file will be generated by Protool. Thus layout does not matter. It will
juste be cut\&paste.
\section*{A1 Administrative Overview}
\subsection*{Summary}
\begin{tabular}{|c|c|}
\hline
Proposal Acronym &  \\ \hline
Proposal Number &  \\ \hline
Thematic Priorities & IST-2001-4.3.3. \\ \hline
Call identifier & \\ \hline
Proposal Date & 16 10 2001 \\ \hline
Proposal full name & \\ \hline
\end{tabular}
\subsection*{Abstract (1000 characters)}
\textit{The proposal abstract should be a very short and precise presentation
of the main features of the proposal. Why is it proposed and what problem is it
solving? What are the objectives? How will the objectives be achieved? What
results are expected? This proposal abstract will be used together with the
proposal summary description in form A2 in the evaluation process and in
communications about the proposals to the interested parties (Commission
services and programme committees, etc.). Please use plain typed text, avoiding
formulae and other special characters. If the proposal is written in a language
other than English, please include in form A1 an English version of the
abstract.} \\ \\ \\
In almost two years, the biggest barriers for Free Software development have
been raised by the hosting facilities which provide most of the mandatory tools
that free software developers need for everyday tasks. But although there are
many hosting facilities around the world, no software package is designed to
setup and maintain a set of machines dedicated to this purpose. The NAME
project will provide a generic solution that can be used and adapted by all of
them. Furthermore it will focus on the ease of installation which will fix the
biggest drawback of the current solutions.
\\ \\ 
586 characters
\begin{tabular}{|c|c|}
\hline
Duration in months & \\ \hline
Keyword & BESTP \\ \hline
\end{tabular}
\section*{A2 Proposal Summary}
\textit{This form, should be filled in by the co-ordinator only, for single
participant proposals by the principal contractor. You should not use more than
3,500 characters. The proposal summary should, at a glance, provide the reader
with a clear understanding of the proposal objectives and how the objectives
will be achieved, and their relevance in the context of the objectives of the
specific programme. This summary may be used as an alternative to the proposal
abstract, as the description of the proposal in the evaluation process and in
communications to the programme management committees and other interested
parties. It must therefore be short and precise. Please use plain typed text,
avoiding formulae and other special characters. If the proposal is written in a
language other than English, please include in form A2 an English version of
the proposal summary.} \\


\subsection*{Objectives (max 1000 characters)}
to be written
\subsection*{Description of the work (max 2000 characters)}
to be written
\subsection*{Milestones and expected result (max 500 characters)}

\end{document}
