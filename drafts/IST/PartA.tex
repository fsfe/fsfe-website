\documentclass[a4paper,11pt]{report}
\usepackage[latin1]{inputenc}
\usepackage{graphics}
\begin{document}
\section*{Notice}
This file will be generated by Protool. Thus layout does not matter. It will
juste be cut\&paste.
\section*{A1 Administrative Overview}
\subsection*{Summary}
\begin{tabular}{|c|c|}
\hline
Proposal Acronym &  SICHA \\ \hline
Proposal Number &  ??????? \\ \hline
Thematic Priorities & IST-2001-4.3.3. \\ \hline
Call identifier & IST-01-7-1B\\ \hline
Proposal Date & 16 10 2001 \\ \hline
Proposal full name & Simple International Cooperative Hosting Architecture \\ \hline
\end{tabular}
\subsection*{Abstract (1000 characters)}
\textit{The proposal abstract should be a very short and precise presentation
of the main features of the proposal. Why is it proposed and what problem is it
solving? What are the objectives? How will the objectives be achieved? What
results are expected? This proposal abstract will be used together with the
proposal summary description in form A2 in the evaluation process and in
communications about the proposals to the interested parties (Commission
services and programme committees, etc.). Please use plain typed text, avoiding
formulae and other special characters. If the proposal is written in a language
other than English, please include in form A1 an English version of the
abstract.} \\ \\ \\
In almost two years, the biggest barriers for Free Software development have
been raised by hosting facilities which provide most of the mandatory tools
that free software developers need for everyday tasks such as Bug tracking,
mailing lists, Project management... But although there are many hosting
facilities around the world, no software package is designed to setup and
maintain a set of machines dedicated to this purpose. \\ 
The SICHA project will provide a generic solution that can be used and adapted
by all of them.  SICHA will of course provide the same services as current
facilities but will focus an the ease of configuration and installation that
all current ones lack. \\
SICHA will also provide support for internationalization to help dissemination
and adoption in European countries for non english speakers.\\ Finally SICHA
will introduce support for cooperation in hosting platforms, that will be a
first step to the creation of networks of hosting facilities. 
\\ \\ 
989 characters... (846 non white) \\
\begin{tabular}{|c|c|}
\hline
Duration in months & ?????? \\ \hline
Keyword & BESTP \\ \hline
\end{tabular}
\section*{A2 Proposal Summary}
\textit{This form, should be filled in by the co-ordinator only, for single
participant proposals by the principal contractor. You should not use more than
3,500 characters. The proposal summary should, at a glance, provide the reader
with a clear understanding of the proposal objectives and how the objectives
will be achieved, and their relevance in the context of the objectives of the
specific programme. This summary may be used as an alternative to the proposal
abstract, as the description of the proposal in the evaluation process and in
communications to the programme management committees and other interested
parties. It must therefore be short and precise. Please use plain typed text,
avoiding formulae and other special characters. If the proposal is written in a
language other than English, please include in form A2 an English version of
the proposal summary.} \\
\subsection*{Objectives (max 1000 characters)}
The SICHA project aims at developping a complete hosting facility. It is 
composed of two parts. \\
The first one will be a next generation easy to install and international
hosting facility. It will provide a web frontend to a number of tools now 
mandatory for software development (user/groups rights management, bug tracking, mailing lists and forums...). 
The main idea between its design is to make it transparently distributed.
At each step of its development, we will focus on the ease of installation and configuration.  will also provide support for internationalization to help dissemination and adoption in European countries for non english speakers.\\
The second one will provide a generic description of
an hosting facility. With that tool we want to allow different kinds of facility
to communicate with each other. It will help to develop tools to facilitate
migrations between hosting facilities. This part is more oriented on data formats and tools to exchange informations between hosting facilities. It may
be released as an individual package from the first part.\\
\\ 
%1087/924 chars need some check and some other stuff
\subsection*{Description of the work (max 2000 characters)} to be written
%SAVANNAH
The hosting facility is composed of two main parts, the frontend and the
backend. The first one is the code base that interacts with the user and that
will implement services that are not provided by external software. The backend
will link the Database (filled by the frontend) to external software which
provide services such as DNS, mailing lists or CVS. Concurrently the packaging
of the solution will allow the validation and testing of each milestone.\\
Our primary goal is to build a set of packages that is easy to
install on a virgin machine or a machine already used for other
purposes. 
Although we have in mind that the set of packages produced could be used for
multiple purposes, our focus will be on development hosting facilities. \\

To achieve this goal, we will 
1 redesing existing hosting architectures such as Savannah and Picolibre 
  so that they may be packaged, installed and taylored easily; 
2 integrate multilanguage support to these platforms and implement support
  versions for portuguese, german, spanish, english and french;
3 define common data types and representation between the two types of
  platform with possible interaction with other platform via the CoopX project;
4 introduce functionalities to import/export informations from one site to
  another, based on this common data representation;
5 test and deploy over wide area network at least three sites in three out
	of four participating countries.
%1445/1224 chars need check and more description
\subsection*{Milestones and expected result (max 500 characters)}
PM 3: specification of new hosting architecture
PM 6: specification of common data formats
PM 9: V1 of savannah with new design 1 language version
PM12: V1 of picolibre with new design
      V2 of savannah with support for at least 2 languages and import/export
PM15: V2 of picolibre with multilingual support for at least 2 languages and
      import/export functionalities
PM18: deployment of redesigned platform in 3 countries, 
      dissemination of the project results by a Debian package.
%496/408
\section*{A4 Cost Summary}
%% will look at that later
\textbf{This must be discussed very soon}
\rotatebox{90}{
\begin{tabular}{c|c|c|c|c|c|c|c|c|c|c|c|c|c|c|c|c|c|c|}
Participant Role(CO/P) & Participant Number & Linked to Contractor No &
Participant Short Name & Number of Person - Months & Personnel Costs &
Durable Equipment & Consumables & Travel and subsitence & Computing &
External assistance & Other Costs & Knowledge protection on Costs & 
Overhead Costs & Total Costs & Requested Contribution from the Commision &
Eligibility (Y/No)\\
\end{tabular}
}
\end{document}
