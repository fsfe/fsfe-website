%% $Id: Constitution.en.tex,v 1.1 2001-07-17 13:16:16 taz Exp $
%% 
%% Constitution of the Free Software Foundation Europe
%%
%% Language: English
%%

%{{{ Header

\documentclass[a4wide,12pt]{article}
\usepackage{german,umlaut}
\usepackage{fancyheadings}
\usepackage{alltt}
\usepackage[german]{babel}

\selectlanguage{german}
\pagestyle{fancyplain}

\lhead[\fancyplain{}{\bfseries\thepage}]
        {\fancyplain{}{\bfseries\rightmark}}
\rhead[\fancyplain{}{\bfseries\leftmark}]
        {\fancyplain{}{\bfseries\thepage}}
\cfoot{}
\renewcommand{\sectionmark}[1]{\markright{\thesection\ #1}}
\renewcommand{\thesection}{\S~\arabic{section}}

%
% alternative ways of doing it...
%

% this makes the references work correctly:
%\newcommand{\A}{\subsection{}}
%\renewcommand{\subsectionmark}[1]{\markright{\thesection\ }}

% this looks better:
\newcounter{absatz}[section]
\newcommand{\A}{\par\vspace{1ex}
                \stepcounter{absatz}\noindent(\arabic{absatz})~~}

%}}}

\begin{document}

%{{{ 0 Preamble

\thispagestyle{empty}
\begin{center}
\Huge\bf Constitution
\end{center}

\section*{Preamble}

Upon entering the digital age, in which real and virtual space will
equally determine the social, cultural and scientific development of
mankind, the Free Software Foundation Europe has the long-term goal to
raise and work on the questions this will necessarily raise.\\

In this regard the direct function is the unselfish promotion of
Free Software as well as creating and propagating the awareness of the
related philosophical and social questions.\\

As its acknowledged sister organization, the FSF Europe will join
forces with the Free Software Foundation founded by Richard
M. Stallman in the United States of America. The latter, recognized
tax-exempt charitable organization in the USA, has been dedicating
itself since 1984 to the promotion and distribution of Free software
and in particular the GNU-System, a Unix-like operating system. This
system is mostly known by one of its variants, GNU/Linux, which since
1993 has been used successfully on many computers.\\

The term Free Software in the sense of the FSF Europe does not refer
to the price, but rather to the following four freedoms:
\begin{enumerate}
\item freedom: the freedom to use a program for any purpose
\item freedom: the freedom to study the program and adapt it to your
own needs. 
\item freedom: the freedom to make copies for others.
\item freedom: the freedom to improve a program and make these
improvements available to others, so that the whole community
benefits. 
\end{enumerate}

This definition of Free Software goes back to the idea of freely
exchanging knowledge and ideas that can traditionally be found in
scientific fields. Like thoughts, software is non-tangible and
duplicable without loss. Passing feeds an evolutionary process,
advancing thoughts and software.\\

Only Free Software preserves the possibility to comprehend and build
upon scientific results. For scientists, it is the only kind of
software which corresponds to the ideals of a free
science. Accordingly, the promotion of free software is also a
promotion of science.\\

The distribution of information and the forming of an opinion are done
increasingly by digital media, and the trend is to foster the use of
those means for a direct citizen participation to
democracy. Therefore, a central task of the FSF Europe is to train
proficient citizens in these media, thereby promoting democracy.\\

Digital space (``Cyberspace''), with software as its medium and its
language has an enormous potential for the promotion of all mental and
cultural aspects of mankind. By making it commonly available and
opening up the medium, Free Software grants equal chances and
protection of privacy.\\

Coining the awareness for the problems related to the digital age in
all parts of society is long-term goal and a core aspect of the work
of the FSF Europe.\\

Therefore the FSF Europe will seek to increase the use of Free
Software in schools and universities in order to parallelize the
education in real space matters with the creation of understanding and
awareness of problems in virtual space.\\

Free Software guarantees traceable results and decision-making
processes in science and public life as well as the individual rights
to free development of personality and liberty of opinion. It is the
job of the FSF Europe to carry Free Software into all areas that touch
public life or ``informational human rights'' of citizens.\\

%}}}

%{{{ 1 Name, seat, financial year

\section{Name, seat, financial year} 
\A 
The association bears the name ``Free Software Foundation Europe.''  It
is to be registered into the register of associations; after the
registration it leads the additive ``e.V..''
\A 
The association has its seat in Hamburg.  
\A
The financial year is the calendar year.

%}}}

%{{{ 2 Purpose, tasks, non-profit character

\section{Purpose, tasks, non-profit character}
\A 
Purpose of the FSF Europe is the promotion of the goals specified in
the preamble.
\A
The goals of the FSF Europe are namely to be achieved by:
\begin{enumerate}
\item the support of governmental and private organisations in all
aspects of the Free Software, 
\item the cooperation and coordination of the national associations
which pursue the same goals 
\item the support of projects working on Free Software,
\item the distribution of the philosophical ideals of Free Software  
\end{enumerate}
\A 
The FSF Europe pursues exclusively and directly publicly-spirited
and scientific purposes in the sense of the section ``tax-privileged
purposes'' of the tax code.  The FSF Europe is working selflessly and
does not pursue self-economic goals.

\A 
Means of the FSF Europe may be used for the statutory purposes only.
No person may be favored by expenses alien to the goals of the FSF
Europe or disproportionally high.

%}}}

%{{{ 3 Acquisition of membership 

\section{Acquisition of membership}

\A 
Any national or foreign natural or legal person may become member of
the FSF Europe. Natural persons must be 16 years old.  Persons under
age do not have eligibility for election.
\A 
Condition for the acquisition of the membership is a written
application for membership to the executive committee.  
\A 
The general assembly of the members decides on the application
for membership with three quarters of all members, that are natural
people.  The executive committee can grant the application for
membership passing;  the application for membership must then be
confirmed by the next members assembly. In case of refusal of the
request no obligation exists to communicate the reasons to the
applicant.

%}}}

%{{{ 4 End of membership

\section{End of membership}
\A 
Membership ends 
\renewcommand{\labelenumi}{\alph{enumi}.}
\begin{enumerate}
\item with the death of the member with natural persons and/or its
liquidation in the case of legal persons;
\item by withdrawal from the association;
\item by exclusion from the association.  
\end{enumerate}
\renewcommand{\labelenumi}{\arabic{enumi}.}
\A
The withdrawal is made via written declaration vis-a-vis the executive
committee. The declaration withdrawal must be signed by the competent
legal representative. The withdrawal can be declared at any time.
\A
For important reasons or if the bond of trust between the members is
broken, a member can be excluded by decision of the executive
committee from the association. Before deciding on the matter, the
executive committee must give the member opportunity to state its
position in writing. The decision of the executive committee is to be
justified in writing and sent to the member. The member can appeal the
decision at the general assembly of the members.  The appeal must be
lodged within two weeks after communication of the decision at the
executive committee. The executive committee has to call in a general
assembly within three months of punctual insertion of the appointment,
which can support the decision of the executive committee with a
majority of three quarters of the remaining members.  Until the final
decision about the exclusion, the member remains suspended of all
obligations and all rights.

%}}}

%{{{ 5 Membership fees

\section{Membership fees}
The members make contributions by ways of honorary activity or through
holding a position in the association.

%}}}

%{{{ 6 Structure of the association

\section{Structure of the association}
\A 
The Free Software Foundation Europe forms a European federation
structure and is divided into national associations. Those are
associations with own juridical personality after the law on societies
of the European State for its area the association are active.  All
members of the national associations must also be members of the FSF
Europe.  
\A 
In order to preserve the uniformity, the national associations have to
fulfill minimum requirements determined by the extended executive
committee, which are written down in a constitution-template for
national associations. This applies with the exception of requirements
that are inadmissible according to the laws of the state in which the
national association is to be created. In this case the constitution
is to be modified so it reflects the intentions of the
constitution-template most closely. The constitution requires the
acceptance of the extended executive committee before the national
association becomes part of the FSF Europe.
\A
The finances of the national associations are determined by the
financial order in accordance with \S~9 sec. 2 of this
statute.
\A 
The national associations can conclude contracts in their own name for
the fulfillment of local tasks, if the means are present for the
fulfilment of these contracts.  They can in no case enter obligations
for the FSF Europe.
\A
The national associations can only enter negotiations with authorities
and organizations mainly active in their respective areas.
\A
The national associations are entitled and obliged to bear their own
name trough which the affiliation with the Free Software Foundation
Europe is expressed. This has to be done by adding the denomination
``Chapter'' with the name of the state in English to ``Free Software
Foundation Europe.'' Additionally they can bear the name in the
national language.
\A
The executive committee is entitled to extract the right for the
guidance of the name ``FSF Europe'' from a national association that
violates this constitution or its own constitution. The national
association can appeal the decision at the general assembly of the
members.  The appeal must be lodged within two weeks after
communication of the decision at the executive committee. The
executive committee has to call in a general assembly within three
months of punctual insertion of the appointment, which can remedy or
overrule the decision of the executive committee with a majority of
three quarters of the remaining members.  Until the final decision
about the exclusion, the member remains suspended of all obligations
and all rights.

%}}}

%{{{ 7 Bodies of the association

\section{Bodies of the association}
The Free software Foundation Europe forms a European federation
composed by the association of 7 organs/bodies. Organs of the FSF
Europe are:
\renewcommand{\labelenumi}{\alph{enumi})}
\begin{enumerate}
\item the executive committee (President);
\item the Vice-President;
\item the administrative director (Head of Office);
\item the extended executive committee;
\item the general assembly.
\end{enumerate}
\renewcommand{\labelenumi}{\arabic{enumi}.}

%}}}

%{{{ 8 The executive committee 

\section{The executive committee}
The executive committee of the FSF Europe consists of the President.

%}}}

%{{{ 9 The competence of the executive committee 

\section{The competence of the executive committee}
\A
The executive committee is responsibly for all affairs the FSF Europe,
as far as they are not transferred to another organ of the FSF Europe
by statute.  It has in particular the following tasks:
\begin{enumerate}
\item Convening and preparation of the general assembly of the members
as well as list the agenda;
\item Creation of the annual report;
\item Adoption of resolutions over the admission and the exclusion of
members;
\item Support and control of the national associations;
\item Agency in judicial and except-court affairs, in particular also
authority traffic; 
\item Relationships with the press.
\end{enumerate}
\A
The executive committee leads the finances after the finance plan,
which the general assembly with a majority of three quarters of all
voices decides on.  The finance plan must give,
\begin{enumerate}
\item that possible profits are used only for the statutory purposes
\item that no member may get no shares of the profits or other
allowances from the means of the FSF Europe or its local
associations. This also applies to separating members.
\item that administrative costs, which are alien to the purpose of the
FSF Europe may not be granted. Same applies to disproportionately high
payments.
\end{enumerate}
\A
The executive committee is not liable vis-a-vis the association
for slightly negligent behaviour.

%}}}

%{{{ 10 Election and term of office of the executive committee 

\section{Election and term of office of the executive committee}
\A
The executive committee is elected by the general assembly of the
members for the duration of two years, from the election on. It
continues to be in office up to the new election of the executive
committee, however.
\A
Electable are only members of the FSF Europe. If the executive
committee separates prematurely, the vice-president represents the
association up to the selection of a new executive
committee. Additionally, an extraordinary general assembly of the
members is to be called in within three months by the vice-president. 

%}}}

%{{{ 11 The Vice-President

\section{The Vice-President}
\A
The vice-president represents the executive comittee in the following
cases:
\begin{enumerate}
\item Separating of the executive committee;
\item Passing indispensability of the executive committee.
\end{enumerate}
\A 
The executive committee is indispensable, if he communicates this in
written form to the vice-president. The vice-president takes care of
all matters as long and to the extent they were transferred to him in
writing by the executive committee. The executive committee is
considered indispensable, if it is not attainable or cannot exercise
its office more than seven days because of illness.
\A
The vice-president is elected by the general assembly for the duration
of two years, from the election on.  He remains in office up to the
election of the new vice-president, however. Only members of the FSF
Europe can be elected for vice-president.  With the end of membership
in the association, the office of the vice-president also ends.
\A
If the vice-president is ruled out prematurely, the executive
committee can select a successor for the remaining term of office.
\A
The vice-president is not liable vis-a-vis the association for
slightly negligent behaviour.

%}}}

%{{{ 12 The Head of Office

\section{The Head of Office}
\A
The Head of Office leads the office of the FSF Europe. He represents
the association in the business area belonging to it.
\A
The Head of Office certifies the keys for encoding digital texts and
documents. 
\A
The Head of Office is elected by the general assembly for the duration
of two years, from election on. He remains in office up to the
election of the new Head of Office, however. Only members of the FSF
Europe can be selected for Head of Office. With the end of membership
in the association, the office of the administration director also
ends.
\A
If the administration director is ruled out prematurely, the executive
committee can select a successor for the remaining term of office.
\A
The administration director is not liable vis-a-vis the association
for slightly negligent behaviour.

%}}}

%{{{ 13 The extended executive committee

\section{The extended executive committee}
\A
The extended executive committee consists of the President, the
Vice-President and the Head of Office.  
\A
The extended committee is resolutionable, if at least two members are
present in person, among them the President. Decisions are passed by
single majority of votes.
\A
Members of the extended executive committee are not liable vis-a-vis
the association for slightly negligent behaviour.

%}}}

%{{{ 14 The competence of the extended executive committee 

\section{The competence of the extended executive committee}
The extended executive committee is responsible for the following
tasks:
\begin{enumerate}
\item The approval of the statutes of the national associations;
\item The execution of the ``Guidelines,'' which are decided on by the
general assembly.
\end{enumerate}

%}}}

%{{{ 15 The general assembly

\section{The general assembly}\label{sec:ga}
\A 
In the general assembly, each member that is a natural person, has a
voice. For the practice of the right to vote another member can be
authorized by written message to the executive committee. The
authorisation is to be given for each general assembly separately.  A
member cannot practise the right to vote for more than one third of
all members.
\A
The general assembly has exclusive jurisdiction for the following
affairs:
\begin{enumerate}
\item Creation of so-called ``Guidelines,'' that are to lead the
activity of the executive committee and extended executive committee;
\item Receipt of the annual report of the executive committee;
\item Exoneration of the organs;
\item Choice and recall of the executive committee, the Vice-President
and of the Head of Office.
\end{enumerate}

%}}}

%{{{ 16 The convening of the general assembly 

\section{The convening of the general assembly}\label{sec:cga}
\A 
At least once a year, if possible in the first quarter, is the orderly
general assembly to take place; if possible in one of the states, in
which a national association exists. It is called up in writing by the
executive committee under adherence to one period of six weeks under
indication of the agenda. The period begins with day following the
sending of the invitation letter. The invitation letter is to be
considered delivered to the member, if it is addressed to the last
address or e-mail address given by the member in writing. With written
agreement of three quarters of the members, the invitation period can
be shortened to three weeks.
\A
The agenda is derermined by the executive committee. Each member can
request a supplement of the agenda in writing to the executive
committee until at the latest one week before a general assembly. The
executive committee has to announce the supplement at the beginning of
the general assembly. About applications for supplement of the agenda,
which are made at the general assembly, are decided by the general
assembly.

%}}}

%{{{ 17 The extraordinary general assembly 

\section{The extraordinary general assembly}
The executive committee can call up an extraordinary general assembly
at any time.  It must be called up, if at least a quarter of all
members require it in writing under indication of the purpose and
reasons to the executive committee. To the extraordinary meeting of
the members the \ref{sec:ga}, \ref{sec:cga} and \ref{sec:aorga} apply
accordingly. 

%}}}

%{{{ 18 Adoption of resolutions of the general assembly 

\section{Adoption of resolutions of the general assembly}\label{sec:aorga}
\A
The general assembly is led by the executive committee, at his
indespensability by the Vice-President and at the indispensability of
the latter by the Head of Office.
\A
The type of election is determined by the assembly director. The
election must be carried out in writing, if a third of the members
present in person with the election requests this.
\A
The general assembly is not public.
\A
The general assembly is resolutionable, if it was duly called up and
at least one third of all club members is present or represented by
club members present. In the case of decision inability, the executive
committee is obliged to call up a second general assembly with the
same agenda within four weeks; this general assembly will
resolutionable without consideration of the members present. This is
to be referred to in the invitation.
\A
Unless stated otherwise in the statute, the general assembly passes
decisions with simple majority of the delivered valid voices;
abstentions are therefore left out of the consideration. Changes of
the statute require a majority of three quarters of (the delivered
valid) voices, dissolution of the FSF Europe requires four fifths of
the voices of all members. Changing the purpose of the FSF Europe can
only be decided with unanimous agreement of all members. The written
agreement of the members not present in the general assembly can be
explained only within one month vis-a-vis the executive committee.
\A
In the case of elections it is selected who received more than half of
the delivered valid voices.  If nobody received more than half of the
delivered valid voices, a ballot takes place between the two
candidates, who received most of the voices.  Is selected then who
received most of the voices.  In the case of equal number of votes
once a new choice is necessary; if the mood equality continues to
exist, the lot decides.
\A
Over decisions of the general assembly, a protocol is to be led, that
is to be signed by the recording clerk and the assembly director.  It
is to contain the following ascertainments: Place and time of the
assembly, the person of the assembly director and the recording clerk,
the number of members present, the agenda, the individual election
results and the type of the election.  For amendments of the statute
the exact wording is to be given.

%}}}

%{{{ 19 Dissolution of the association 

\section{Dissolution of the association}
\A
The dissolution of the FSF Europe can only be decided in a general
assembly by the majority in accordance to \ref{sec:aorga}.
\A
If the general assembly decides nothing else, the executive committee
is the liquidator entitled to act as substitute.
\A
After end of liquidation, existing funds go to the 
\begin{alltt}
  Free Software Foundation
  59 Temple Place - Suite 330, 
  Boston, MA 02111-1307, USA, 
\end{alltt}
which it has to use directly and exclusively for non-profit purposes.
\A 
The preceding regulations apply accordingly, if the FSF Europe is
dissolved for another reason or loses its legal capacity.

%}}}

%{{{ 20 Written form

\section{Written form}
Writing form is fulfilled if one of the following conditions is met:
\renewcommand{\labelenumi}{\alph{enumi})}
\begin{enumerate}
\item handwritten signed paper document;
\item E-mail signed with a key that is sufficiently state of the art.
The association decides what is to be regarded as state of the art and
the key must be certified by the association.
\end{enumerate}
\renewcommand{\labelenumi}{\arabic{enumi}.}

%}}}

%{{{ 21 Place of jurisdiction 

\section{Place of jurisdiction}
Place of jurisdiction for all rights and duties resulting from this
statute is Hamburg, Germany.

%}}}

\end{document}