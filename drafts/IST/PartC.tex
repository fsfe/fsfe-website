\documentclass[a4paper,11pt]{report}
\usepackage[latin1]{inputenc}
\usepackage{vmargin}
 
%% Marges
\setmarginsrb{2cm}{1cm}{2cm}{2cm}{1cm}{1cm}{1cm}{1cm}
%1 est la marge gauche
%2 est la marge en haut
%3 est la marge droite
%4 est la marge en bas
%5 fixe la hauteur de l'ent�te
%6 fixe la distance entre l'ent�te et le texte
%7 fixe la hauteur du pied de page
%8 fixe la distance entre le texte et le pied de page
\setlength{\marginparwidth}{0cm}
\setlength{\marginparsep}{0cm}
%\setcounter{tocdepth}{1}

%% XXX
\title{Proposal full title \\ Proposal Acronym}
\date{16/10/2001}
\begin{document}
\chapter*{Notice}
\begin{verbatim}
IST2001 - IV.3.3 Free software development: towards critical mass 
Objectives:

1.To foster in Europe a critical mass of development of free software released
  under GPL-compatible licenses. 
2.To make available European based support services for free software projects.

Focus:

1.Support services for free software developers at all stages of the project 
  lifecycle: project hosting certification, release, repositories, dissemination. 
2.Trials of free software development in the following fields: media technology 
  for personal users,co-operative information production and sharing, and usability. 
3.Socio-economic studies on technology assessment, economic assessment and derived 
  business models 

Types of actions addressed: Trials and studies Accompanying Measures. 
Links with WP 2000: Further focused Action Lines IST 2000 - IV.3.4 and IV.3.5 

more info : http://www.cordis.lu/ist/bwp_en5.htm
\end{verbatim}
\maketitle
\tableofcontents
\chapter*{C3. Community added value and contribution to EC policies}
\addcontentsline{toc}{chapter}{C3. Community added value and contribution to EC policies}
\textit{This section, which should not exceed two pages, should identify which
issue at the European level the proposal is addressing. It should also describe
how the experience will be transferred to other organisations across Europe. \\
This section must include a description of the European dimension of their
action, in terms of profile of the wider community which will benefit from the
result of their action. This will help define the European level issue
addressed by the proposal. \\
Participants should be prepared to be grouped under a Support Node or Cluster,
to ensure that those not directly involved can benefit from the results. In
this way, the Take-Up Actions will generate European added value, beyond the
limit achievable through individual projects.}
\\ 
\section{European dimension of the problem}
A big part of free or open software development is done in Europe. As an 
example see http://www.debian.org/devel/developers.loc, that gives an overview
of the Debian developers community. There are as more developpers in the 
western part of europe than in the rest of the world.
This community, as others in Europe, uses
hosting platforms in Europe as well as in the rest of the world, mostly of
which are in the United States.
\\
For example, the Savannah hosting architecture is deployed on one machine
located in Boston Massachusset, this machine is vital to the GNU developpers
community and it is replicated using rsync techniques in Europe, although 
the replicating machines have the same informations as the original machine
Europeen developpers are obliged to use the original machine. This situtation
is a threat to the data safety, accessibility and security. Let's imagine that
the building where the computer that is hosting savannah burn, how long 
will it take to set up a new plant usable in the same way that the current
machine is. Nobody knows, even if the data are replicated, the procedures to
switch projects from one machine to another do not exist. Nobody is able to 
guess if it is possible or how long will it take.
\\
Having many hosting facilities divide the global risk brought by a centralized 
approach. Having this hosting facilities in area close to the developpers ones
is a good practice in terms of network usage.
\\
Having hosting facilities designed to support multiligual usage will allow a 
greater number of developpers to start developping using their national 
language, lowering the barrier that prevents them from sharing their code.
\\
\section{European added value of the consortium}
The multinational structure of the consortium will help the consortium being
concerned with multilingual support in the development and contribute to
start the platforms with support for four european languages.
\\
The links between the consortium and the GNU project, many members of the 
consortium are also members of the GNU project,
will foster the dissemination of the project results and help the project
in having a real based installed and used by existing projects.
\\
To ensure the SICHA project dissemination it will be packaged so that 
it may be included in existing
distribution non commercial or commercial.
\\
The presence of an industrial partner that will propose services such as
installing, tayloring, and dimensionning hosting platforms for 
industrial purpose, will favour installation of such platforms in industrial
environment to help entreprises in adopting this kind of technologie with a 
reduced access cost. The easy installation features of the product will also
be relevent for industrial usage.
\\
\section{Contribution to EU policies}
The choice of licensing the results of the project under the GNU GPL warrants
that the results will remain available to anyone in the future.
\\
Contributing to adoption of good practice in software development at low
cost, will help improve global software production in europe as well as in
the rest of the world.
\\
Having a hosting platform specific for high education will
contribute to help teaching the best practices for software development in
every university at no extra licensing cost.
\\
Building a network of platforms will help EU in increasing the number of 
developpers that will act for free or open software development. 
\\
The tools used in development platforms may also be used for other kinds of 
groupware activities such as cooperative writing. Providing such tools 
easily instalable and taylorable may help EU to create new networks of 
contributors on different topics such as education (cooperative writing of
courses), helping EU in highering the global level of education by creating
less cost effectives documents.

\chapter*{C4. Contribution to Community social objectives}
\addcontentsline{toc}{chapter}{C4. Contribution to Community social objectives}
\textit{This section, not exceeding two pages, should describe how the proposed
subject wil contribute to meeting the social objectives of the Community, and
should focus on the contribution of the project to improving employment
prospects and the use and development of skills in Europe. \\
Where aplicable the descriptions should also cover how the proposed measure
contributes to improving quality of life and health and safety (including
working condition), and/or to preserving or enhancing the environment and
natural ressources. This might include the contribution of the proposed work to
meeting relevant regulatory requirements.}
\\ \\
The SICHA project will help in creating more free/open software in EU by easing 
the access to a software hosting project platform.
\\
The software created using the hosting platforms, resulting from the SICHA 
project, will help reduce the software costs for individuals as well as for
small companies and large industrial groups. 
\\
The SICHA project will also strengthen existing software by allowing more 
developpers and testers to access hosting platforms and bug reports facilities.
\\
By allowing people to work collectively on projects using their native 
speaking language,
the SICHA project will help the documentation translation and generaly the 
software documentation, that will facilitate a larger adoption of free/open
software in the mass of computer users that do not speak english.
\\

\chapter*{C5. Management}
\addcontentsline{toc}{chapter}{C5. Management}
\textit{This section, not exceeding two pages, should describe how the Take-Up
Action will be managed, the decision making structures to be applied, the
communication flow within the consortium and the quality assurance measures
which will be implemented, and how legal and ethical obligations, for example
concerning encryption or security of personal data, will be met.
\\
The clustering of Assesment project will allow effective co-ordination, early
agreement - between users and suppliers - on globally accepted assesment
metrics and on competitive specifications, the timely delivery of relevant and
measurable results of assesment and their dissemination to other users
worldwide.\\
Actions management of an action is linked to the management of the underlying
business case (target application, baseline project. Relations and dependencies
including the risks involved should also be described. Beging linked under a
Support Node or Cluster will require co-operation with other actions.}
\\ \\
to be written.
\chapter*{C6. Description of the consortium}
\addcontentsline{toc}{chapter}{C6. Description of the consortium}
\textit{One page description of the consortium stating whom the participants
are, what their roles and functions in the consortium are, and how they
complement each other.\\
Where subcontractors are involved they should be clearly identified, and the
scope and nature of their work needs to be justified}
\\ \\ not yet written
\chapter*{C7. Description of the participants}
\addcontentsline{toc}{chapter}{C7. Descriptions of the participants}
\textit{Short description of the participating organisations including (no more
than two pages per organisation):
The expertise and experience of the organisation,
Short CVs of the key persones to be involved indicating relevant exprerience,
expertise and involvement in other EC projects. (Each CV no more than 10
lines). \\
This section should include a brief description of the main business of the
proposer(s) (or consortium, if applicable), including their business motivation
for the proposed action, and the business relevance. \\
It should be clear from this section what the user's current practices are,
which improvements are introduced during the course of the action, and what the
required and expected business practice upon completion of the action is}.
\\ \\ 
%% ALCOVE DE
\section*{Daniel Riek - Alcove Deutschland CTO}
%% Need to be improved and must be shorter
Daniel Riek switched to GNU/Linux in late 1993 short after graduating
from school. In early 1997 during his studies of computing science at the
university of Bonn he founded the Linux service company ID-PRO together
with some friends. During the next years he worked as a developer and
engineer, later he served in the management area of ID-PRO. After ID-PRO
received a major venture capital financing, Daniel became the chairman
and speaker of the board of directors. There he was responsible for the
technical strategy, representation, the community and public relationship.
He also was, together with the companies CTO, responsible for the technical
support and the software development activities of the company.
Caused by the capital market crisis in late 2000 ID-PRO failed to acquire
new funds, that would have been necessary due to the venture capital driven
strategy. After a short rest Daniel joined Alcove to continue his work in
the Free Software business. Daniel is a strong supporter of the Free
Software Foundation Europe and is a member of the board of germany's Linux
vendors association Linux-Verband LIVE e.V.
\section*{GET}
Made up of the six major Graduate Schools of France in the field of Information
Technology, the Groupe des Ecoles de Telecommunications (GET) represents more 
than 600 researchers in 60 research labs, 600 PhD, 2500 engineering and 
management students, 300 master's students. 
The GET is a major higher education group in France and one of the top level 
university in the IT field in Europe.

The vocation of the Groupe des Ecoles des T�l�communications is to provide 
graduate programs in engineering and management tailored to meet these 
challenges in IT field.

As well as demanding academic program the six schools require their students 
to spend time in industry or companies. This hands on experience is to give 
the students first hand knowledge of corporate life and to help them to 
better approach their first employment.

The strategy of the Groupe des Ecoles des T�l�communications to create 
partnerships with companies world-wide is intended to help students become 
an active part of the global market.

The expanding international world of business requires professionals with 
finely honed personal communications and cross cultural management skills. 
As a result, the six graduate schools have traditionally been committed 
to an effective program in foreign languages and social sciences.

Students are required to go abroad to carry out either an internship or an 
academic program.

The GET works closely with the corporate world through its activities in 
research, continuing education teaching and internship programs and also 
welcomes the membership of corporate management for its steering committees. 
Businesses implicated in this way provide valuable feedback so that it may 
best serve the evolving needs of the business community.

\subsection*{PeCoVall group}
The PeCoVall group is composed of five professors and researchers 
from the three institutions ENST, ENST Bretagne and INT.

This group has been formed last year to set up a project under GPL on an 
hosting environment specialy designed for high education. 
The members of the group are much concerned with free/open software and 
during the last three years, they organised five conferences related to 
free/open software in their institutions. One member is also a Debian 
maintener. 

The telecommunication background of the group, and the need of some common 
data between the three sites led them to start a reflexion on cooperative
exchange in hosting platforms.  \\
\subsection*{Dr. Christian Bac: GET coordinator}
Dr. Bac obtained his Bsc in Computer Science from Bordeaux I university.
He then started to work for France Telecommunication Research and 
Development.
Concurently, he obtained his MSc in Computer Science fron Paris
VI University in 1984. 
He began to work on UNIX kernel in 1984 and on the Chorus distributed 
system in 1986.
 
He moved to the "Institut National des Telecommunications"
in 1989, as an Assistant Professor
where he teaches UNIX at various level (user, system calls,
administration, internals), programming languages C, C++, JAVA and
distributed computing (distributed applications, distributed
operating systems).

His research on distributed systems and operating systems were collected
in a Ph.D. document and he recieved a Ph.D in Computer Science from 
the University of Paris XI-Orsay in 1995.
 
He switched his research theme to free software two years ago and sat up a
group in INT focussing on helping people work with GNU/Linux and other related
free software. With researchers from the three biggest schools of GET, they
started a research/development project on hosting platforms focussing on 
high education needs.

Next contributors :  Jaime.
\chapter*{C8. Economic development and scientific and technological prospects}
\addcontentsline{toc}{chapter}{C8. Economic development and scientific and
technological prospecs}
\textit{This section, which should not exceed three pages, should describe
plans for the dissemination and - if relevant - the exploitation of the results
for the consortium as a whole and for the individual participants in contrete
terms. \\
This section must include an explanation of the potential for - internal and
external - replication, redeployment and/or reuse of results of the action. It
should a) describe in detail how the anticipated results will be used
internally within the participating organisations and what decisions may be
taken upon successful completion of the action, and b) identify the business
communities, sharing the problem to be solved by the action and describe a plan
how the results achieved, benefits obtained and lessons learnt (including key
competencies acquired, deficiencies removed, lasting changes) will be
disseminated/communicated and transferred to those communities. \\
Given the real benefits of an action may only become apparent several months
after completion of the action, proposers should indicate expected benefits in
the longer term (including measures that will be taken to assess them)}
\\ \\
to be written
\end{document}
